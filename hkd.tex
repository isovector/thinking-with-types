\documentclass[book.tex]{subfiles}
\begin{document}

\chapter{Higher Kinded Data}

\section{What is Higher Kinded Data?}

\preamble{Database}

Sometimes when writing code, we need several variations on the same type.
For sake of discussion, let's assume we're writing a relational database purely
in Haskell. Such a thing will require at a minimum, a type corresponding to a
database table, and a type corresponding to a row in said table. We might also
want a type for describing atomic updates to the table.

One approach might be to model our row and our table as separate types.

\snip{Database}{PersonRow}
\snip{Database}{PersonTable}

The reasoning behind using \ty{Seq} at \ann{1} but \ty{IntMap} at \ann{2} is to
differentiate between types which must not have empty fields, and ones which
may. Our choices (especially of \ty{Seq}) might not make much sense in a real
application, but the goal is simply to motivate having differing storage
back-ends.

This approach is somewhat dissatisfying aesthetically; it requires a duplication
of effort whenever the database schema changes---one must remember to change
both types.

Instead, a more pleasing approach is defining our table with the  so-called
\defn{higher-kinded data} (or \defn{HKD}) technique. Higher-kinded data is the
practice of parameterizing a datatype, and then making each of its fields a type
family which uses that parameter. The idea is that by fiddling the parameters to
the datatype, we can use the type family to build very different representations
of the same ``pattern''.


\section{An HKD Database}

To demonstrate, we can use the technique by factoring a \defn{HKD pattern} out
of our \ty{PersonRow} and \ty{PersonTable}. We can parameterize
\ty{PersonPattern} on a data kind corresponding to which \ty{Representation} of
our pattern we want. Should it be a table, a row, or an atomic update? At the
same time, we'll add semantic meaning to our earlier \ty{Maybe}
type---indicating that it can be \ty{'Nullable}, rather than specifying a
concrete implementation of what that looks like.

Given the following data kinds:

\snip{Database}{Representation}
\snip{Database}{Nullable}

We can write our person pattern like this:

\snip{Database}{PersonPattern}

Notice that \hs{name}, \hs{age}, and \hs{address} are all now values of
\ty{Column r}. By making \ty{Column} a closed type family, we're able to pull
very different instantiations of \ty{PersonPattern} simply by changing our
\kind{Representation} \ty{r}.

The actual implementation of \ty{Column} will depend on the use case, but for
our example it'll be implemented thusly:

\snip{Database}{Column}

\ty{Column} is organized along two dimensions---its \kind{Representation}, and
whether or not it's \kind{Nullable}. \kind{Nullable} types have some notion of
partiality in their resultant type, whether it be an \ty{IntMap t} (with
possibly missing keys) for its table representation, or a \ty{Maybe t} as its
row. The \ty{Update} type will be defined momentarily, but corresponds to either
updating a value or leaving it be. At \ann{1}, we wrap \ty{Update} in a
\ty{Maybe}, in order to also support \emph{unsetting} fields.

\snip{Database}{Update}

While higher-kinded data is good for tying multiple, similar types together via
a pattern definition---it requires the help of \ty{GHC.Generics} to get anything
done between the types.

For example, it seems reasonable to want to be able to locate a particular row
inside of our table. It's trivial to write \hs{lookup :: Int -> PersonPattern
'TableRep -> Maybe (PersonPattern 'RowRep)} (try it!), but doing it in general
for any pattern is trickier.

We begin with a carrier typeclass:

\snip{Database}{GLookup}

And we can provide instances for \ty{K1} over both \ty{Column 'TableRep 'NotNull
a} and \ty{Column 'TableRep 'Null a}. While we'd like to write the instances
directly in terms of \ty{Column}, GHC unfortunately won't allow us to give class
instances for type families. As such, we'll need to expand the definitions
ourselves.

\snip{Database}{GLookupK1Seq}
\snip{Database}{GLookupK1IntMap}

The other generic instances are trivial and boilerplate.

\snip{Database}{GLookupU1}
\snip{Database}{GLookupV1}
\snip{Database}{GLookupTimes}
\snip{Database}{GLookupM1}

Our desired \hs{lookup} function can now be defined:

\snip{Database}{dumbLookup}

While this works, we'll notice that every function we write in terms of HKD will
require the same sort of constraints. It'll save us time to write a constraint
synonym (\apageref{constraint synonym}) and use that instead. Doing so will
require \ext{ConstraintKinds}, \ext{UndecidableSuperClasses} and
\ext{UndecidableInstances}, because we're describing a superclass context as a
polymorphic \kind{Constraint} \ty{c}.

\snip{Database}{HKDImpl}
\snip{Database}{InstHKDImpl}

Given these, we can redefine \hs{lookup} in terms of \ty{HKDImpl}.

\snip{Database}{lookup}

Finding things in an empty database isn't particularly useful, so let's add the
ability to \hs{insert} a row.

\snip{Database}{GInsert}

Writing base-case instances for \ty{GInsert} requires inserting a \ty{Column
'RowRep n a} into a \ty{Column 'TableRep n a}. Because there is no way to get
the ``next index'' in an \ty{IntMap}, \ty{GInsert} requires receiving the
desired index of the inserted row. It is assumed some external machinery will be
responsible for generating these keys and maintaining their uniqueness.

\snip{Database}{GInsertK1IntMap}
\snip{Database}{GInsertK1Seq}

\begin{exercise}
Complete the generic implementation of \ty{GInsert} by implementing
the rest of the generic instances.
\end{exercise}
\begin{solution}
\end{solution}

We can use GHCi to test our work.

\begin{dorepl}{Database}
let db = PersonPattern S.empty S.empty IM.empty
let db' = insert 0 (PersonPattern "Sandy" 27 Nothing) db
lookup 0 db'
let db'' = update 0 (PersonPattern Keep (Set 28) (Just Keep)) db'
lookup 0 db''
\end{dorepl}

However, initializing an empty database (\hs{db} in the above example) is
tedious and boilerplate.

\snip{Database}{GEmpty}
\snip{Database}{empty}

\begin{exercise}
Implement all necessary instances of \ty{GEmpty}.
\end{exercise}
\begin{solution}
The interesting instances are:
\snip{Database}{K1GEmpty1}
\snip{Database}{K1GEmpty2}
\end{solution}

\begin{exercise}
Use \ty{GHC.Generics} to implement \ty{update :: Int -> pat 'UpdateRep
-> pat 'TableRep -> pat 'TableRep}.
\end{exercise}
\begin{solution}
\end{solution}


\section{Limitations}

The HKD pattern has two major drawbacks which are important to keep in mind.

Most damningly, because HKD's performance relies on heavy usage of
\ty{GHC.Generics}, its use can balloon compile-times by an order of magnitude or
so. In effect, you're trading fast developer time and runtime for slow compile
times. Thankfully, this can be alleviated in most cases by only running the
relevant simplifier passes for production builds. The
\ty{GHC.Generics}-unoptimized code will run fast enough for local testing, while
if resources permit, the heavy work of optimizing away generics can be left to
your build farm.

The other potential problem you'll run into is that HKD doesn't play nicely with
GHC's deriving mechanisms. For example, GHC will refuse to derive an \ty{Eq}
instance for us:

\begin{dorepl}{Database}
:set -XStandaloneDeriving
deriving instance Eq (PersonPattern f)
\end{dorepl}

This is a frustrating limitation of type families in GHC---as they're
implemented today, the compiler is completely unwilling to insert the necessary
instance constraints for us. Unfortunately, not even the
\ext{QuantifiedConstraints} extension---which provides quantifiers \emph{inside}
constraints---is capable of helping us here.

The solution is simple, if unsavory. All that is necessary is to enable
\ext{StandaloneDeriving} and put concrete types on our deriving statements.

\snip{Database}{deriving}

\end{document}

