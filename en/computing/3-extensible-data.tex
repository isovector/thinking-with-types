\documentclass[book.tex]{subfiles}
\begin{document}

\chapter{Extensible Data}

\section{Introduction}

One heralded feature of dynamic languages---conspicuously missing in
Haskell---is their support for ad-hoc objects. Consider Python's dictionary
datatype:

\begin{alltt}
record = \{ 'foo': 12, 'bar': True \} \annotate{1}
record['qux'] = "hello" \annotate{2}
record['foo'] = [1, 2, 3] \annotate{3}
\end{alltt}

At \ann{1}, \texttt{record} can be said to have a (Haskell) type
\hs{\{foo :: Int, bar :: Bool\}}. However, at \ann{2} it also gains a
\hs{qux :: String} field. And at \ann{3}, the type of \hs{foo} changes to
\hs{[Int]}. Of course, Python doesn't actually enforce any of these typing
rules. It is, however, a good example of a program that would be hard to
represent in Haskell.

In this chapter, we will discuss how to build this sort of ``extensible'' record
type. As its dual, we will also investigate generalizing the \ty{Either} type to
support an arbitrary number of potential types. These types are a good and
instructive use of type-level programming.

When looking at the canonical representations of types, we saw that all of
custom Haskell data-types can be built as \emph{sums-of-products}. Sums
correspond to \ty{Either a b}, and products correspond to the pair type \ty{(a,
b)}. However, just because we can show an \gls{isomorphism} between a type and this
sum-of-products doesn't mean such an encoding is \emph{efficient}.

Because both \ty{Either} and \ty{(,)} are \emph{binary operators},
using them for arbitrarily large representations will require lots of extra
constructors. In the best case our representation will be a balanced binary tree
(of the form \ty{Either (Either ...) (Either ...)}.) This corresponds to an
overhead of $O(\log{n})$ constructors. In the worst case, it degrades into a
linked list, and we pay for $O(n)$ constructors.

It seems, however, like we should be able to describe a sum of any number of
possibilities in $O(1)$ space, since it only ever contains one thing. Likewise,
that we can describe a product in $O(n)$ space (rather than the $O(n+\log{n})$
balanced binary tree, or $O(2n)$ linked list.)

The ``trick'' building extensible sum and product types is to use the
type system to encode the extra information. Anything we do at the type-level is
free at runtime. Because open sums have fewer moving pieces, we will work
through them first, and use their lessons to help build open products.


\section{Open Sums}

\preamble{OpenSum}

By definition, an open sum is a container of a data whose type isn't known
statically. Furthermore, there are no guarantees that we know which types it
\emph{might be,} since the list of types itself might be polymorphic.

Existential types are ones whose type has been forgotten by the type system. As a
result, we can use them to allow us to store \emph{any} type inside of our open
sum container. We will constrain this later on.

Although they're not necessary, as we've seen, GADTs provide a nice interface
for defining and working with existential types.

\snip{OpenSum}{OpenSum}

At \ann{1} we declare \ty{OpenSum} as a GADT. It has two parameters, \ty{f}
of kind \ty{k -> Type} and \ty{ts} of kind \kind{[k]}. We call \ty{f} an
\emph{indexed type}, which means it provides a \kind{Type} when given a
\kind{k}. We will talk more about these parameters in a minute.

The data constructor \hs{UnsafeOpenSum} at \ann{2} is thus named because, well,
its unsafe. We'll later provide tools for constructing \ty{OpenSum}s safely, but
building one via the data constructor isn't guaranteed to maintain our
invariants. It's a common pattern in type level programming to label raw data
constructors as \hs{Unsafe}, and write smart constructors that enforce the
safety.

Looking at \ann{3}, we see that \ty{OpenSum} is a container of \ty{f t}, where
\ty{t} has kind \kind{k}. Because \ty{t} isn't mentioned in the return type of
our constructor (\ann{4}), \ty{t} is existential. Once \ty{t} enters an
\ty{OpenSum}, it can never be recovered. Knowledge of what \ty{t} is is lost
forever.

Returning to our parameters, we assign the semantics that the existential \ty{t}
must be one of the elements of \ty{ts}. If \ty{ts \tyeqSpace '[ Int, Bool, String ]},
we know \ty{t} was originally one of \ty{Int}, \ty{Bool} or \ty{String}, though
we do not necessarily know which.

You might be wondering about the value of the \ty{f} parameter. Its presence
allows us to add a common shape to all the members of \ty{ts}. For example,
\ty{OpenSum ((->) String) '[Int, Bool]} is capable of storing \ty{String -> Int}
and \ty{String -> Bool}.

Users who just want to store regular types with no additional structure should
let \ty{f \tyeqSpace Identity}.

The \ty{OpenSum} data constructor also stores an \ty{Int}, which we will use
later as a tag to ``remember'' what type the existential variable \ty{t} had. It
will be used as an index into \ty{ts}. For example, if we are storing the number
\hs{2} and \ty{ts \tyeqSpace '[A, B, C, D]}, we understand that originally, \ty{t
\tyeqSpace C}.

A first-class type family can be used to find \ty{t} in \ty{ts}:

\snip{OpenSum}{FindElem}

\ty{FindElem} works by looking through \ty{ts} and comparing the first element
of each tuple with \ty{key}. The result of \ty{FindIndex} is a \kind{Maybe k},
which we then, at \ann{1}, call the type-level equivalent of \hs{fromMaybe} on.
If \ty{FindIndex} returned \ty{'Nothing}, \ty{FindElem} returns \ty{Stuck}. A
stuck type family is one which can't reduce any further. We can exploit this
property and ask whether \ty{FindElem} evaluated fully by checking whether it's
a \ty{KnownNat}.

\snip{OpenSum}{Member}

A benefit of this approach is that a \ty{KnownNat} constraint allows for
reification of the \kind{Nat} at the term-level---we can get an \ty{Int}
corresponding to the \kind{Nat} we had. Additionally, the type-level nature of
\ty{FindElem} means we pay no runtime cost for the computation.

\snip{OpenSum}{findElem}

Armed with \hs{findElem}, we are able to build a smart, type safe constructor
for \ty{OpenSum}s.

\snip{OpenSum}{inj}

\hs{inj} allows injecting a \ty{f t} into any \ty{OpenSum f ts} so long as
\ty{t} is an element somewhere in \ty{ts}. However, nothing about this
definition suggests \ty{ts} must be monomorphic; it can remain as a type
variable so long as we propagate the \ty{Member t ts} constraint upwards.

But a sum type is no good without the ability to try to take things out of it.

\snip{OpenSum}{prj}

Projections out of \ty{OpenSum} are done by a runtime check at \ann{1} that the
\ty{Int} type tag inside of \ty{OpenSum} is the same as the type we're trying to
extract it as. If they are the same, at \ann{2} we know it's safe to perform an
\hs{unsafeCoerce}, convincing the type checker to give back a (non-existential)
\ty{t}. If the type tags are not the same, \hs{prj} gives back \hs{Nothing}.

We find ourselves wanting to reduce an \ty{OpenSum} regardless of whats inside
it. \ty{prj} is no help here---it either succeeds or it doesn't. If it fails, we
the programmers have learned something about the type inside (what it is
\emph{not}). The types don't reflect that knowledge.

\snip{OpenSum}{decompose}

The \hs{decompose} function lets us perform induction over an \ty{OpenSum}.
Either we will get out the type corresponding to the head of \ty{ts}, or we will
get back a \ty{OpenSum} with fewer possibilities. A type tag of 0 corresponds
to the head of the type list, so it is unnecessary to call \hs{findElem}. We
maintain this invariant by decrementing the type tag in the \hs{Right} case.

In practice, it is also useful to be able to \emph{widen} the possibilities of
an open sum. A new function, \hs{weaken}, tacks a \ty{x} type in front of
the list of possibilities.

\begin{exercise}
Write \hs{weaken :: OpenSum f ts -> OpenSum f (x ': ts)}
\end{exercise}
\begin{solution}
\unspacedSnip{OpenSum}{weaken}
\end{solution}

If we want to perform the same logic regardless of what's inside an
\ty{OpenSum}, \hs{prj} and \hs{decompose} both feel inelegant. We introduce
\hs{match} eliminator which consumes an \ty{OpenSum} in $O(1)$ time.

\snip{OpenSum}{match}

By using a rank-n type at \ann{1}, \hs{match} is given a function that can
provide a \ty{b} \emph{regardless of what's inside the sum.} In this case,
inspecting the type tag isn't necessary.

There is a general principle to take away here. If it's too hard to do at the
type-level, it's OK to cheat and prove things at the term-level. In these cases,
\hs{unsafeCoerce} can be your best friend---so long as you're careful to hide
the unsafe constructors.


\section{Open Products}

\preamble{OpenProduct}

Open products are significantly more involved than their sum duals. Their
implementation is made sticky due to the sheer number of moving pieces. Not only
do open products have several internal types it's necessary to keep track of,
they also require a human-friendly interface.

Our implementation will associate names with each type inside. These names can
later be used by the user to refer back to the data contained. As a result, the
majority of our implementation will be type-level book-keeping. Inside the
product itself will be nothing new of interest.

We begin by defining a container \ty{Any} that will existentialize away its
\ty{k} index. \ty{Any} is just a name around the same pattern we did in
\ty{OpenSum}.

\snip{OpenProduct}{Any}

This implementation of \ty{OpenProduct} will optimize for $O(1)$ reads, and
$O(n)$ writes, although other trade-offs are possible. We thus define
\ty{OpenProduct} as a \ty{Data.Vector} of \ty{Any}s.

\snip{OpenProduct}{OpenProduct}

\ty{OpenProduct} is structured similarly to \ty{OpenSum}, but its \ty{ts}
parameter is of kind \kind{[(Symbol, k)]}. The parameter \ty{ts} now pulls
double duty---not only does it keep track of which types are stored in our
\ty{Vector Any}, but it also associates names to those pieces of data. We will
use these \kind{Symbol}s to allow the user to provide his own names for the
contents of the product.

Creating an empty \ty{OpenProduct} is now possible. It has an empty \ty{Vector}
and an empty list of types.

\snip{OpenProduct}{nil}

Because all data inside an \ty{OpenProduct} will be labeled by a \kind{Symbol},
we need a way for users to talk about \kind{Symbol}s at the term-level.

\snip{OpenProduct}{Key}

By way of \extension{TypeApplications}, \ty{Key} allows users to provide labels
for data. For example, \hs{Key @"myData"} is a value whose type is \ty{Key
"myData"}. Later \apageref{OverloadedLabels} we will look at how to
lessen the syntax to build \ty{Key}s. These are the necessary tools to insert
data into an open product. Given a \ty{Key key} and a \ty{f k}, we can insert a
\ty{'(key, k)} into our \ty{OpenProduct}.

\snipRename{OpenProduct}{badInsert}{insert}

Our function \hs{insert} adds our new \ty{'(key, k)} to the head of the type
list, and inserts the \ty{f k} to the head of the internal \ty{Vector}. In this
way, it preserves the invariant that our list of types has the same ordering as
the data in the \ty{Vector}.

We can test this in GHCi with the \ghcicmd{:t} command.

\begin{dorepl}{OpenProduct}
/badInsert/insert/let result = badInsert (Key @"key") (Just "hello") nil
:t result
/badInsert/insert/:t badInsert (Key @"another") (Just True) result
\end{dorepl}

While this looks good, there is a flaw in our implementation.

\begin{dorepl}{OpenProduct}
@let result = badInsert (Key @"key") (Just "hello") nil
/badInsert/insert/:t badInsert (Key @"key") (Just True) result
\end{dorepl}

Trying to \hs{insert} a duplicate key into an \ty{OpenProduct} succeeds. While
this isn't necessarily a \emph{bug}, it's confusing behavior for the user
because only one piece of keyed data will be available to them. We can fix this
with a type family which computes whether a key would be unique.

\snip{OpenProduct}{UniqueKey}

\ty{UniqueKey} is the type-level equivalent of \hs{null . filter (== key) .
fst}. If the \ty{key} doesn't exist in \ty{ts}, \ty{UniqueKey} returns
\ty{'True}. We can now fix the implementation of \hs{insert} by adding a
constraint to it that \ty{UniqueKey key ts \tyeqSpace 'True}.

\snipRename{OpenProduct}{oldInsert}{insert}

GHCi agrees that this fixes the bug.

\begin{dorepl}{OpenProduct}
@let result = oldInsert (Key @"key") (Just "hello") nil
/oldInsert/insert/:t oldInsert (Key @"key") (Just True) result
\end{dorepl}

Informative it is not, but at least it fixes the bug. In the next chapter, we
will look at how to make this error message much friendlier.

To project data out from an open product, we'll first need to write a getter.
This requires doing a lookup in our list of types to figure out which index of
the \ty{Vector} to return. The implementation is very similar to that for
\ty{OpenSum}, except that we compare on the key names instead of the \kind{k}s
themselves.

\snip{OpenProduct}{FindElem}

\snip{OpenProduct}{findElem}

We will also require another type family to index into \ty{ts} and determine
what type to return from \hs{get}. \ty{LookupType} returns the \kind{k}
associated with the \ty{key}ed \kind{Symbol}.

\snip{OpenProduct}{LookupType}

\snip{OpenProduct}{get}

At \ann{1}, we say the return type of \hs{get} is \ty{f} indexed by the
result of \ty{LookupType key ts}. Since we've been careful in maintaining our
invariant that the types wrapped in our \ty{Vector} correspond exactly with
those in \ty{ts}, we know it's safe to \hs{unsafeCoerce} at \ann{2}.

As one last example for open products, let's add the ability to modify the value
at a key. There is no constraint that the new value has the same type as the old
one. As usual, we begin with a first-class type family that will compute our new
associated type list. \ty{UpdateElem} scans through \ty{ts} and sets the type
associated with \ty{key} to \ty{t}.

\snip{OpenProduct}{UpdateElem}

The implementation of \hs{update} is rather predictable; we update the value
stored in our \ty{Vector} at the same place we want to replace the type in
\ty{ts}.

\snip{OpenProduct}{update}

\begin{exercise}
Implement \hs{delete} for \ty{OpenProduct}s.
\end{exercise}
\begin{solution}
\unspacedSnip{OpenProduct}{delete}
\end{solution}

\begin{exercise}
Implement \hs{upsert} (update or insert) for \ty{OpenProduct}s.

  Hint: write a type family to compute a \kind{Maybe Nat} corresponding to the
  index of the key in the list of types, if it exists. Use class instances to
  lower this kind to the term-level, and then pattern match on it to implement
  \hs{upsert}.
\end{exercise}
\begin{solution}
This is a particularly involved exercise. We begin by writing a FCF to compute
  the resultant type of the \hs{upsert}:

\snip{OpenProduct}{UpsertElem}

  Notice that at \ann{1} we refer to \ty{Placeholder1Of3}---which is a little
  hack to get around the lack of type-level lambdas in FCFs. Its definition is
  this:

\snip{OpenProduct}{Placeholder1Of3}
\snip{OpenProduct}{EvalPlaceholder}

  The actual implementation of \hs{upsert} requires knowing whether we should
  insert or update. We will need to compute a \kind{Maybe Nat} for the type in
  question:

\snip{OpenProduct}{UpsertLoc}

  And we can use a typeclass to lower this the \kind{Maybe Nat} into a \ty{Maybe
  Int}:

\snip{OpenProduct}{FindUpsertElem}
\snip{OpenProduct}{FindUpsertNothing}
\snip{OpenProduct}{FindUpsertJust}

  Finally, we're capable of writing \hs{upsert}:

\snip{OpenProduct}{upsert}
\end{solution}


\section{Overloaded Labels}
\label{OverloadedLabels}

We earlier promised to revisit the syntax behind \ty{Key}. Working with
\ty{OpenProducts} doesn't yet bring us joy, mostly due to the syntactic noise
behind constructing \ty{Key}s. Consider \hs{get (Key @"example") foo}; there are
nine bytes of boilerplate syntactic overhead. While this doesn't seem like a
lot, it weighs on the potential users of our library. You'd be surprised by how
often things like these cause users to reach for lighter-weight alternatives.

Thankfully, there \emph{is} a lighter-weight alternative. They're known as
\defn{overloaded labels}, and can turn our earlier snippet into \hs{get
\#example foo}.

Overloaded labels are enabled by turning on \ext{OverloadedLabels}. This
extension gives us access to the \hs{\#foo} syntax, which gets desugared as
\hs{fromLabel @"foo" :: a} and asks the type system to solve a \ty{IsLabel "foo"
a} constraint. Therefore, all we need to do is provide an instance of
\ty{IsLabel} for \ty{Key}.

\snip{OpenProduct}{keyIsLabel}

Notice that the instance head is \emph{not} of the form
\hs{IsLabel key (Key key)}, but instead has two type variables (\ty{key} and
\ty{key'}) which are then asserted to be the same (\ann{1}). This odd
phrasing is due to a quirk with Haskell's instance resolution, and is known as
the \defn{constraint trick}.

The machinery behind instance resolution is unintuitive. It will only match
\defnn{instance heads}{instance head} (the part that comes after the fat
constraint arrow \hs{=>}). The instance head of \hs{(Eq a, Eq b) => Eq (a, b)}
is \hs{Eq (a, b)}. Once it has matched an instance head, Haskell will work
backwards and only then try to solve the context (\hs{(Eq a, Eq b)} in this
example.) All of this is to say that the context is not considered when matching
looking for typeclass instances.

It is this unintuitive property of instance resolution that makes the constraint
trick necessary. Notice how when we're looking for key \hs{\#foo}, there is
nothing constraining our return type to be \ty{Key "foo"}. Because of this, the
instance Haskell looks for is \hs{IsLabel "foo" (Key a)}.

If our instance definition were of the form \hs{instance IsLabel key (Key key)},
this head \emph{won't} match \hs{IsLabel "foo" (Key a)}, because Haskell has no
guarantees \ty{"foo" \tyeqSpace a}. Perhaps we can reason that this must be the case,
because that is the only relevant instance of \ty{IsLabel}---but again, Haskell
has no guarantees that someone won't later provide a different, non-overlapping
instance.

By using the constraint trick, an instance head of the form \hs{IsLabel key (Key
key')} allows Haskell to find this instance when looking for \ty{IsLabel "foo"
(Key a)}. It unifies \ty{key \tyeqSpace "foo"} and \ty{key' \tyeqSpace a}, and then will
expand the context of our instance. When it does, it learns that \ty{key \tyeq
key'}, and finally that \ty{a \tyeqSpace "foo"}. It's roundabout, but it gets there
in the end.

The definition of \ty{IsLabel} can be found in \ty{GHC.OverloadedLabels}.

\end{document}
